\documentclass[lang=cn,newtx,10pt,scheme=chinese]{elegantbook}

\usepackage{xpatch}

\ExplSyntaxOn
    \xpatchcmd\__xeCJK_check_family:n{\__xeCJK_warning:nxx}{\__xeCJK_info:nxx}{}{}
\ExplSyntaxOff

% 全局设置
% 中文默认字体:宋体,
\setCJKmainfont{FandolSong}

\title{概率论与数理统计习题}
\subtitle{茆诗松版}

\author{Mick Huang}

\setcounter{tocdepth}{3}

\cover{cover.jpg}

% 本文档命令
\usepackage{array}
\usepackage{framed}
\usepackage{mathrsfs}
\newcommand{\ccr}[1]{\makecell{{\color{#1}\rule{1cm}{1cm}}}}

% 修改标题页的橙色带
\definecolor{customcolor}{RGB}{32,178,170}
\colorlet{coverlinecolor}{customcolor}
\usepackage{cprotect}

\addbibresource[location=local]{reference.bib} % 参考文献,不要删除

\begin{document}

\maketitle
\frontmatter

\tableofcontents

\mainmatter

\chapter{随机事件与概率}
\section{随机事件及其运算}

\begin{theorem}[事件运算性质]
    1.\ 交换律:
    \begin{equation}
        A \cup B = B \cup A
    \end{equation}
    \begin{equation}
        AB = BA
    \end{equation}
    2.\ 结合律
    \begin{equation}
        (A \cup B) \cup C = A \cup (B \cup C)
    \end{equation}
    \begin{equation}
        (AB)C = A(BC)
    \end{equation}
    3.\ 分配律:
    \begin{equation}
        (A \cup B) \cap C = AC \cup BC
    \end{equation}
    \begin{equation}
        (A \cap B) \cup C = (A \cup C) \cap (B \cup C)
    \end{equation}
    4.\ 对偶律(德摩根公式):
    \begin{equation}
        \overline{A \cup B} = \bar{A} \cap \bar{B}
    \end{equation}
    \begin{equation}
        \overline{A \cap B} = \bar{A} \cup \bar{B}
    \end{equation}
    \begin{equation}
        \overline{\bigcup_{i=1}^{n}A_i} = \bigcap_{i=1}^{n}{\bar{A_i}}
    \end{equation}
    \begin{equation}
        \overline{\bigcup_{i=1}^{\infty}A_i} = \bigcap_{i=1}^{\infty}{\bar{A_i}}
    \end{equation}
    \begin{equation}
        \overline{\bigcap_{i=1}^{n}A_i} = \bigcup_{i=1}^{n}{\bar{A_i}}
    \end{equation}
    \begin{equation}
        \overline{\bigcap_{i=1}^{\infty}A_i} = \bigcup_{i=1}^{\infty}{\bar{A_i}}
    \end{equation}
    5.\ 差公式:
    \begin{equation}
        A-B=A\bar{B}
    \end{equation}
\end{theorem}
\newpage

\begin{problemset}[习题 1.1]
    \item 写出下列随机试验的样本空间:
    \item[(1)]抛三枚硬币;
    \item[(2)]抛三枚骰子;
    \item[(3)]连续抛一枚硬币,直至出现正面为止;
    \item[(4)]口袋中有黑、白、红球各一,从中任取2个球,先从中取1,放回去后再取1;
    \item[(5)]口袋中有黑、白、红球各一,从中任取2个球,先从中取1,不放回再取1。
    \begin{solution}
    \begin{framed}
        \item[(1)] $\Omega=\left\{ZZZ,ZZF,ZFZ,ZFF,FZZ,FZF,FFZ,FFF\right\}$
        \item[(2)] $\Omega=\left\{x,y,z \in (1,2,3,4,5,6)\ |\ (x,y,z)\right\}$ 
        \item[(3)] $\Omega=\left\{(Z),(F,Z),(F,F,Z),...,(F,F,F,...,F,Z)\right\}$
        \item[(4)] $\Omega=\left\{(B,B),(B,W),(B,R),(W,B),(W,W),(W,R),(R,B),(R,W),(R,R)\right\}$
        \item[(5)] $\Omega=\left\{(B,W),(B,R),(W,B),(W,R),(R,B),(R,W)\right\}$
    \end{framed}
    \end{solution}

    \item 先抛一枚硬币,若出现正面(记为$Z$),则再掷一枚骰子,试验停止;若出现反面(记为$F$),则再抛一次硬币,试验停止。那么,该试验的样本空间$\Omega$是什么?
    \begin{solution}
        \begin{framed}
            $\Omega=\left\{X \in (0,1,2,3,4,5,6)\ |\ (Z,X),(F,F),(F,Z)\right\}$
        \end{framed}
    \end{solution}

    \item 设$A,B,C$为三事件,试表示下列事件:
    \item[(1)] $A,B,C$都发生或都不发生;
    \item[(2)] $A,B,C$中不多于一个发生:
    \item[(3)] $A,B,C$中不多于两个发生;
    \item[(4)] $A,B,C$中至少有两个发生。
    \begin{solution}
        \begin{framed}
            \item[(1)] $ABC \cup \bar{A}\bar{B}\bar{C}$
            \item[(2)] $\bar{A}BC \cup A\bar{B}C \cup AB\bar{C} \cup \bar{A}\bar{B}\bar{C}$ 
            \item[(3)] $\overline{ABC}$
            \item[(4)] $ABC \cup \bar{A}\bar{B}\bar{C}$
        \end{framed}
    \end{solution}

    \item 指出下列事件等式成立的条件。
    \item[1] $A \cup B = A$;
    \item[2] $AB=A$。
    \begin{solution}
        \begin{framed}
            \item[1] $B \subset A$
            \item[2] $A \subset B$
        \end{framed}
    \end{solution}
    \newpage

    \item 设$X$为随机变量,其样本空间为$\Omega=\left\{0 \leq X \leq 2\right\}$,记事件$A=\left\{0.5 < X \leq 1\right\}$,$B=\left\{0.25 \leq X < 1.5\right\}$,写出下列各事件:
    \item[(1)] $\bar{A}B$;
    \item[(2)] $\bar{A} \cup B$;
    \item[(3)] $\overline{AB}$;
    \item[(4)] $\overline{A \cup B}$
    \begin{solution}
        \begin{framed}
            \item[(1)] $\bar A = \left\{0 \leq X \leq 0.5\right\} \cup \left\{ 1 < X \leq 2\right\} \Rightarrow \bar{A}B=\left\{0.25 \leq X 0.5\right\} \cup \left\{ 1 < X < 1.5\right\}$
            \item[(2)] $\bar{A} \cup B = \Omega$
            \item[(3)] $AB = A \Rightarrow \overline{AB} = \bar{A} = \left\{0 \leq X \leq 0.5\right\} \cup \left\{ 1 < X \leq 2\right\}$
            \item[(4)] $\overline{A \cup B}=\bar{B}=\left\{0 \leq X < 0.25\right\} \cup \left\{1.5 \leq X \leq 2\right\}$
        \end{framed}
    \end{solution}

    \item  检查三件产品,只区分每件产品是合格品(记为0)与不合格品(记为1),设$X$为三件产品中的不合格品数,指出下列事件所含的样本点:\\
    $A=``X=1",\ B=``X>2",\ C=``X=0",\ D=``X=4"$.
    \begin{solution}
        \begin{framed}
            $A=\left\{(1,0,0),(0,1,0),(0,0,1)\right\}\\
            B=\left\{(1,1,1)\right\}\\
            C=\left\{(0,0,0)\right\}\\
            D=\emptyset$
        \end{framed}
    \end{solution}

    \item 试问下列命题是否成立?
    \item[(1)] $A-(B-C)=(A-B)\cup C$
    \item[(2)] 若 $AB=\emptyset$且$C \subset A$,则$BC=\emptyset$
    \item[(3)] $(A \cup B)-B=A$
    \item[(4)] $(A-B)\cup B=A$
    \begin{solution}
        \begin{framed}
            \item[(1)] $A-(B-C)=A-B\bar{C}=A\overline{B\bar{C}}=A(\bar{B}\cup C)=A\bar{B} \cup AC=(A-B) \cup AC \neq (A-B) \cup C$\quad 不成立
            \item[(2)] 成立
            \item[(3)] $(A \cup B)-B=(A \cup B)\bar{B}=A \bar{B} \cup B \bar{B} = A \bar{B} \neq A$\quad 不成立
            \item[(4)] $(A-B)\cup B=(A \bar{B}) \cup B = (A \cup B) \cap (\bar{B} \cup B)=A \cup B \neq A$\quad 不成立
        \end{framed}
    \end{solution}

    \item 若事件$ABC=\emptyset$,是否一定有$AB=\emptyset$?
    \begin{solution}
        \begin{framed}
            不一定,有可能$AB \neq \emptyset$但$AB \cap C = \emptyset$;或者$A、B、C$两两互不相交。
        \end{framed}
    \end{solution}
    \newpage

    \item 请叙述下列事件的对立事件:
    \item[(1)] $A=$\ ``\ 掷两枚硬币,皆为正面\ ''\ ;
    \item[(2)] $B=$\ ``\ 射击三次,皆命中目标\ ''\ ;
    \item[(3)] $A=$\ ``\ 加工四个零件,至少有一个合格品\ ''\ 。
    \begin{solution}
        \begin{framed}
            \item[(1)] $\bar A=$\ ``\ 掷两枚硬币,最多只有一枚为正面\ ''\ ;
            \item[(2)] $\bar B=$\ ``\ 射击三次,没有全部命中目标\ ''\ ;
            \item[(3)] $\bar C=$\ ``\ 加工四个零件,全部是不合格品\ ''\ 。
        \end{framed}
    \end{solution}

    \item 证明下列事件的运算公式:
    \item[(1)] $A=AB \cup A\bar{B}$
    \item[(2)] $A \cup B = A \cup \bar{A}B$
    \begin{proof}
        \begin{framed}
            \item[(1)] $AB \cup A\bar{B}=A(B \cup \bar{B}) = A$
            \item[(2)] $A \cup B=A\cup BA \cup B\bar{A} = A \cup \bar{A}B$
        \end{framed}
    \end{proof}

    \item 设$\mathscr{F}$为一事件域,若$A_n \in \mathscr{F}\ , n=1,2,...$,试证:
    \item[(1)] $\emptyset \in \mathscr{F}$;
    \item[(2)] 有限并$\bigcup\limits_{i=1}^{n}{A_i} \in \mathscr{F}, n \geq 1$;
    \item[(3)] 有限交$\bigcap\limits_{i=1}^{n}{A_i} \in \mathscr{F}, n \geq 1$;
    \item[(4)] 可列交$\bigcap\limits_{i=1}^{\infty}{A_i} \in \mathscr{F}$;
    \item[(5)] 差运算$A_1 - A_2 \in \mathscr{F}$。
    \begin{proof}
        \begin{framed}
            \item[(1)] 因为$\mathscr{F}$为一事件域,所以$\Omega \in \mathscr{F}$,故其对立事件$\bar{\Omega}=\emptyset \in \mathscr{F}$
            \item[(2)] 因为$A_n \in \mathscr{F}$,所以任一事件满足$X \in \bigcup\limits_{i=1}^{n} \in \mathscr{F}$
            \item[(3)] 因为$A_i \in \mathscr{F}$,所以$\bar{A_i} \in \mathscr{F}$,$\bigcap\limits_{i=1}^{n}A_i = \overline{\bigcup\limits_{i=1}^n{\bar{A_i}}} \in \mathscr{F} \Rightarrow \bigcap\limits_{i=1}^{n}{A_i} \in \mathscr{F}$
            \item[(4)] 因为$A_i \in \mathscr{F}$,所以$\bar{A_i} \in \mathscr{F}$,$\bigcap\limits_{i=1}^{\infty}A_i = \overline{\bigcup\limits_{i=1}^\infty{\bar{A_i}}} \in \mathscr{F} \Rightarrow \bigcap\limits_{i=1}^{\infty}{A_i} \in \mathscr{F}$
            \item[(5)] 因为$A_2 \in \mathscr{F}$,所以$\bar{A_2} \in \mathscr{F}$,由(3)(有限交)得$A_1-A_2=A_1 \cap \bar{A_2} \in \mathscr{F}$。
        \end{framed}
    \end{proof}
\end{problemset}
\newpage

\section{概率得定义及其确定方法}
\begin{axiom}[概率三公理]
    设 $\Omega$ 为一个样本空间。$\mathscr{F}$为$\Omega$得某些子集组成的一个事件域。如果对任一事件$A \in \mathscr{F}$,定义在$\mathscr{F}$上的一个实值函数$P(A)$满足:\newline
    (1)\ 非负性公理\qquad \quad 如果$ A \in \mathscr{F}$,\ 那么$P(A) \ge 0$。\newline
    (2)\ 正则性公理\qquad \quad $P(\Omega)=1$ \newline
    (3)\ 可列可加性公理\quad 如果$A_1,A_2,...,A_n,...$互不相容,那么:\newline
    $$P(\bigcup_{i=1}^{\infty}A_i)=\sum_{i=1}^{\infty}{P(A_i)}$$
    则称$P(A)$为事件$A$得概率,称三元素($\Omega,\ \mathscr{F},\ P$)为概率空间。
\end{axiom}

\begin{problemset}[习题 1.2]
    \item 对于组合数$\dbinom{n}{r}$,证明:\vspace{6pt}
    \item[(1)] $\dbinom{n}{r} = \dbinom{n}{n-r}$;\vspace{6pt}
    \item[(2)] $\dbinom{n}{r} = \dbinom{n-1}{r-1}+\dbinom{n-1}{r}$;\vspace{6pt}
    \item[(3)] $\dbinom{n}{0} + \dbinom{n}{1} + ... \dbinom{n}{n} = 2^n$;\vspace{6pt}
    \item[(4)] $\dbinom{n}{1} + 2\dbinom{n}{2} + ... + n\dbinom{n}{n} = n2^{n-1}$;\vspace{6pt}
    \item[(5)] $\dbinom{a}{0} \dbinom{b}{n} + \dbinom{a}{1} \dbinom{b}{n-1} + ... + \dbinom{a}{n} \dbinom{b}{0} = \dbinom{a+b}{n},\quad n=min(a,b)$;\vspace{6pt}
    \item[(6)] $\dbinom{n}{0}^2 + \dbinom{n}{1}^2 + ... + \dbinom{n}{n}^2 = \dbinom{2n}{n}^2$.\vspace{6pt}
    \vspace{6pt}
    \begin{proof}
        \begin{framed}
            \item[(1)] $\dbinom{n}{r} = \dfrac{\dfrac{n!}{(n-r)!}}{r!} = \dfrac{n!}{r! \cdot (n-r)!}$\vspace{6pt}
            \item[] $\dbinom{n}{n-r} = \dfrac{n!}{(n-r)! \cdot [n-(n-r)]!} = \dfrac{n!}{(n-r)! \cdot r!} = \dbinom{n}{r}$\vspace{6pt}
            \item[(2)] $\dbinom{n-1}{r-1}=\dfrac{(n-1)!}{(r-1)! \cdot [n-1-(r-1)]!}=\dfrac{(n-1)!}{(r-1)! \cdot (n-r)!}$\vspace{6pt}
            \item[] $\dbinom{n-1}{r}=\dfrac{(n-1)!}{r! \cdot (n-1-r)!}$\vspace{6pt}
            \item[] $\dbinom{n-1}{r-1}+\dbinom{n-1}{r}=\cdot \left[\dfrac{(n-1)! \cdot r}{r! \cdot (n-r)!}+\dfrac{(n-1)! \cdot (n-r)}{r! \cdot (n-r)!}\right]=\dfrac{(n-1)! \cdot n}{r! \cdot (n-r)!} = \dfrac{n!}{r! \cdot (n-r)!} =\dbinom{n}{r}$\vspace{6pt}
            \item[(3)] 由二项式定理$(a+b)^n=\sum\limits_{i=0}^{n}{\dbinom{n}{i}a^{n-i}b^{i}}$\vspace{6pt}
            \item[] 令$a=1,\ b=1 \Rightarrow 2^n=(1+1)^n = \sum\limits_{i=0}^{n} \dbinom{n}{i}1^{n-i}1^{i} = \dbinom{n}{0}+\dbinom{n}{1}+...+\dbinom{n}{n}$ \vspace{6pt}
            \item[(4)] $\dbinom{n}{r} = \dfrac{n!}{r! \cdot (n-r)!}=\dfrac{n \cdot (n-1)!}{r \cdot (r-1)! \cdot [(n-1)-(r-1)]!} = \dfrac{n}{r} \cdot \dfrac{(n-1)!}{(r-1)! \cdot [(n-1)-(r-1)]!} $\vspace{6pt}
            \item[] $r\dbinom{n}{r} = r \cdot \dfrac{n}{r} \cdot \dfrac{(n-1)!}{(r-1)! \cdot [(n-1)-(r-1)]!} = n\dbinom{n-1}{r-1}$\vspace{6pt}
            \item[] $\dbinom{n}{1} + 2\dbinom{n}{2} + ... + n\dbinom{n}{n} = n2^{n-1} = n \cdot \dbinom{n-1}{1-1} + n \cdot \dbinom{n-1}{2-1} + ... + n \cdot \dbinom{n-1}{n-1} = n \cdot \sum\limits_{i=0}^{n-1}\dbinom{n-1}{i}$\vspace{6pt}
            \item[] 根据(3),$n\sum\limits_{i=0}^{n-1}\dbinom{n-1}{i}=n2^{n-1}$\vspace{6pt}
            \item[(5)] $ $\vspace{6pt}
            \item[(6)] $ $\vspace{6pt}
        \end{framed}
    \end{proof}

    \item 抛三枚硬币,求至少出现一个正面的概率.
    \begin{solution}
        \begin{framed}
            $A=\{$抛三枚硬币至少出现一个正面$\}$\newline
            $\bar{A}=\{$抛三枚硬币全部都是反面$\} \Rightarrow P(\bar{A})=\dfrac{1}{2^3}=\dfrac{1}{8}$\newline
            $P(A) = 1- P(\bar{A}) = 1-\dfrac{1}{8} = \dfrac{7}{8}$
        \end{framed}
    \end{solution}

    \item 任取两个正整数,求它们的和为偶数的概率.
    \begin{solution}
        \begin{framed}
            因为奇数+奇数=偶数,偶数+偶数=偶数,只有奇数+偶数=奇数\newline\vspace{6pt}
            两个正整数之和的事件空间为$\Omega=\left\{(E,E),(E,O),(O,E),(O,O)\right\}$\newline\vspace{6pt}
            $P(A)=\dfrac{2}{4}=\dfrac{1}{2}$\vspace{6pt}
        \end{framed}
    \end{solution}

    \item 掷两颗骰子,求下列事件的概率:
    \item[(1)] 点数之和为6;
    \item[(2)] 点数之和不超过6;
    \item[(3)] 至少有一个6点。
    \begin{solution}
        \begin{framed}
            两颗骰子掷出的可能性共有$6^2=36$种。
            \item[(1)] 点数之和为6的分别为$A=\left\{(1,5),(2,4),(3,3),(3,3),(4,2),(5,1)\right\}$,共6种。\vspace{6pt}
            \item[] $P(A)=\dfrac{6}{36}=\dfrac{1}{6}$\vspace{6pt}
            \item[(2)] $P(A) = \dfrac{5+4+3+2+1}{36}=\dfrac{15}{36}=\dfrac{5}{12}$\vspace{6pt}
            \item[(3)] $P(A) = 1- \dfrac{5^2}{36} = \dfrac{11}{36}$\vspace{6pt}
        \end{framed}
    \end{solution}

    \item 抛三枚硬币,求至少出现一个正面的概率.
    \begin{solution}
        \begin{framed}
            
        \end{framed}
    \end{solution}

    \item 抛三枚硬币,求至少出现一个正面的概率.
    \begin{solution}
        \begin{framed}
            
        \end{framed}
    \end{solution}

    \item 抛三枚硬币,求至少出现一个正面的概率.
    \begin{solution}
        \begin{framed}
            
        \end{framed}
    \end{solution}
\end{problemset}

\end{document}
